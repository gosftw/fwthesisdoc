\chapter{INTRODUCCIÓN}

\pagenumbering{arabic}

\noindent La combinación de estos módulos logran un XX de porcentaje de detección de incendios forestales con una variedad de imágenes de incendios forestales. blallbalbalb

\section{Descripción del problema}

\section{Objetivos}

\subsection{Objetivo general}
\noindent Desarrollar un sistema de detección de incendios en áreas forestales, mediante el uso de técnicas de Visión por Computador y Aprendizaje Supervisado.

\subsection{Objetivos específicos}

\begin{itemize}
\item[1.] Identificar las características de los incendios mediante técnicas de Visión por Computador.
\item[2.] Modelar la detección de incendios mediante Aprendizaje Supervisado.
\item[3.] Integrar las técnicas de Visión por Computador y los modelos de detección de incendios.
\item[4.] Realizar un análisis de los resultados de la experimentación realizada en una muestra de imágenes.
\end{itemize}

\section{Justificación}

\section{Límites y alcances}

\section{Método de investigación}

\noindent Esta investigación se realiza con un enfoque experimental donde en base al planteamiento del problema, se formula la hipótesis de detección visual de incendios mediante técnicas de Visión por Computador y Aprendizaje Automático.

\noindent El diseño metodológico presentado en la imagen refleja el esquema de trabajo seguido durante todo el desarrollo del presente proyecto de grado.

