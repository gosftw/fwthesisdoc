\chapter*{RESUMEN}
\addcontentsline{toc}{chapter}{RESUMEN}
\markboth{RESUMEN}{RESUMEN}

\noindent Esta tesis describe un sistema para la detección de incendios en áreas forestales que puede ser implementado usando métodos de visión por computadora. El objetivo es crear un sistema que procese una entrada de vídeo que puede estar en un lugar estático o en un drone.

\noindent Se presentan antecedentes de incendios forestales de todo el mundo, y se examina cada situación donde se utilizan capturas de vídeo durante o después de que el desastre se ha producido. Hasta la fecha, los drones no han jugado un papel central en el seguimiento de áreas forestales para detectar anomalías que puedan llegar a generar incendios forestales.

\noindent El propósito del sistema de detección de incendios forestales debe ser capaz de otorgar alerta y ayuda visual para las personas que tienen acceso al sistema. Se espera que esto aumente la capacidad de reacción de las entidades pertinentes para que el incendio sea controlado lo mas pronto posible.

\noindent El sistema de detección consta de 2 módulos, el primero esta centrado en la búsqueda de fuego y el segundo en la detección de humo. %Aca describir los metodos que utilizaran cada modulo HOGDescriptor y un clasificador SVM, etc.

\noindent La combinación de estos módulos logran un XX de porcentaje de detección de incendios forestales con una variedad de imágenes de incendios forestales.
